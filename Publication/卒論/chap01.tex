\chapter{序論}
\section{背景}
画像ノイズ除去はノイズが発生した画像から, ノイズを含まない原画像を推定することを目的としており, 画像処理分野における基礎的なタスクであることから多様な研究が行われている. 
近年, 自動運転や監視カメラ, 医療などの分野において画像認識技術が幅広く活用されており, 今後もその対象は拡がっていくと予想されている前述の画像ノイズ除去は, このような画像認識の認識精度を向上させるための前処理としても重要な役割を担っている.\\
 現在, 画像のノイズ除去においては高いノイズ除去性能を持つ深層学習を用いた手法が主流となっており, 特に深層畳み込みニューラルネットワーク(CNN)が高い性能を示すことが明らかになっている. CNNの欠点としては, 畳み込み層の受容野が比較的狭く,長距離の画素間依存性を認識できないことが挙げられる. このため, 既存のCNNを用いた画像のノイズ除去手法では, 画像の持つコンテクスト(文脈情報)を判断することが困難であり, 除去すべきノイズと残すべき細部の特徴とを区別できないことが知られている.\\
 このようなCNNの問題を解決するための新たな手法として, Transfomer\cite{SA}に代表される自己注意機構を利用することが画像処理の分野で拡がっている\cite{ViT,Swin}. 自己注意機構は自然言語処理の分野で開発された手法で, 離れた位置に存在する単語間の関係を認識することで, 様々な自然言語処理を高精度で行えることが明らかになっている. 長距離間の依存関係を認識できる自己注意機構をノイズ除去に応用することにより, 画像のコンテクストを把握しながらノイズ除去を行うことが可能となり, 高いノイズ除去性能が実現できることが明らかになっている\cite{Swin,Restormer}. 一方で, コンテクスト抽出と(抽出したコンテクストに基づく)ノイズ除去の二つのタスクを異なる二種類のCNNで行うことで効率的にノイズ除去を行えるGTCNNと呼ばれる手法が提案されており\cite{GTCNN}, 画像ノイズ除去タスクにおいてコンテクストの把握とノイズ除去処理のすべてを自己注意機構のみで行うことは, CNNと同様に効率的でない可能性がある.

\section{目的}
前述のようにノイズ除去の二つのタスクを異なる二種類のCNNで行うことで効率的にノイズ除去を行うことができる. このうちコンテクスト抽出タスクを行うCNNをCNNよりコンテクスト把握に有効であるといわれている自己注意機構に置き換えることで, 既存手法と比較してより高いノイズ除去性能または効率的なノイズ除去性能を得ることを目的とする.