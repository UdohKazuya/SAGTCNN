\chapter{結論}
本論文では, GTCNNのコンテクスト抽出を行うGTLに自己注意機構を組み込んだ手法, SAGTCNNを提案した. 自己注意機構を挿入する場所を変更したSAGTCNN-S4, SAGTCNN-M, SAGTCNN-S4Mの3種類のパターンで実験を行った. 実験ではGTCNNに対しSAGTCNNはいくつかのケースでわずかに上回る性能を示した一方で, 明確な優位性を示すことはできなかった. ノイズ除去性能の向上があまり見られなかった理由はいくつか考察できる. まず一般的に自己注意を用いる手法では大規模なデータセットを用いており, 今回の実験で使用した学習用のデータセットでは学習画像枚数が少ない可能性が考えられる. また, 自己注意は広域のコンテクストを捉えることを得意とするが, 今回GTLの4段目という画像サイズの小さい領域でしか自己注意を用いておらず, 本来の性能が活かせなかったことが考えられる. 最後に画像に自己注意を用いる際にViTなどで行われている位置埋め込みを本実験では行っていないため, コンテクストの抽出が上手く行えなかった可能性が考えられる. \\
 今後の課題としては, 位置埋め込みの実装やリアルノイズでの性能比較, 自己注意をGTLの上層にて使用した際の有効性の検討などが挙げられる. 